\documentclass[a4paper,10pt]{report}

\usepackage[english]{babel}

\title{DS - Linc (Linc Is Not Centralized)}
\author{Simon \textsc{Fernandez}}
\date{\today}

\begin{document}
\maketitle

\chapter{Agent deployment}
\section{Topology}
The chosen topology is a random topology. The network must only be 
connected. All algorithms are designed so that they can be applied to
any kind of topology. 

\section{Nodes}
There are two types of nodes in LINC.
\begin{itemize}
\item \textbf{Agents :} They store the data, receive the commands from
users, forward the messages across the topology. They can be linked to
any kind of node.
\item \textbf{Users :} Where the user is. Commands sent by the user are
sent to this node and then sent to the topology. User nodes are entry 
nodes of the topology. Users can only be linked to Agents. Inter-user
links are forbidden.
\end{itemize}

\section{Communications}
Because the topology is random and unknown, sending a message from node
A to node B is not trivial. In LINC I implemented a message passing 
protocol similar to the protocols used in Internet networks or Local 
networks. 

Each node has a list of its immediate neighbours (in the \texttt{nodes()}
variable). When sending a message to an immediate neighbour, the node
just sends the message to the node. But when a node A wants to send a 
message to a distant node B, it must know where to send the message. I 
decided to forbide broadcasting messages not used for routing protocols.
So, in order to find where the target node B is, the node A first 
broadcasts \texttt{\{whereis, B\}}. When a node receives this kind of
message, if it knows how to reach B, it replies with 
\texttt{\{path, By, B, Dist\}} meaning "I (noted By) know how to reach 
B in Dist hops".
When receiving a \texttt{path} message, a node updates a list of 
known distant nodes: now it knows that it can reach B in Dist+1 hops
by sending the message By. 

Each node only keeps the shortest path, and updates it when receiving
\texttt{path} messages.
The paths are stored in an \texttt{ets} table named \texttt{linc\_route}.
In this table there are elements \texttt{\{Target, Distance, Via\}}, 
meaning that in order to reach \texttt{Target}, the node must send the
message to \texttt{Via} and \texttt{Via} will forward it to the next node
etc, etc...

With this protocol, nodes keep a table telling who to send the message
to in order to reach each node. 

\section{Killing a node}
(Data management will be considered in the Data chapter)

When a node is properly killed with the command \texttt{shutdown}, it 
sends to its neighbours a \texttt{dead} message. This way, neighbours
remove the node from their routing tables.

Then, if a node is killed, or dies, we may have the problem of connectivity.
The topology may have been splited in two or more disconnected parts.
This situation is detected with Acknowledgment messages. There is a list
of "important" commands than necessitate an Ack message. So when a node
sends this kind of message, it will wait for an Ack (see next section
for more in-depth description). If the node does not receive the Ack, then
the message may be lost, or the node dead. In this situation, the node
calls the function \texttt{shrink}. This function considers all nodes
reachable in 2 hops, and builds a direct link to these nodes. This way,
if a neighbour died, we add a link to bypass this dead node. 
Once the new path is established, we send the message one more time.

The advandages of this method is that it limits the number of connections
built. But if too many nodes die at the same time, the damages may be 
irrecuperable and part of the topology may be splited.

\section{Ack and how Agents behave}

The first basic principle of Agent behaviour is that Agents must always
be listening for messages, and must NEVER block, waiting for a message.

When receiving a command to execute, the node spawns a process that will
execute the command. But the communication protocol adds some challenge.

\subsection{Sending to distant nodes}
In previous section I described how the \texttt{whereis} message works.
But remember that the node must never be waiting for an answer to its
message. So, when a node A wants to send a message to distant node B, 
it sends the \texttt{whereis} message, and puts the message for B in a 
table named \texttt{linc\_wait}. Messages in this table are waiting for 
path to their recipient. So, each time a node learns a new path by 
receiving a \texttt{path} message, it checks if it has messages to send
to this newly-discovered node. This way, the node is never waiting for
the results of its \texttt{whereis} message.

\subsection{Ack}
Important messages require an Ack message. But we must never wait for 
the Ack. So, when a message is sent, we add an entry in the
\texttt{linc\_ack} table, noting which Ack needs to be received. 
When receiving an Ack, we delete it from the table. And regularly, we 
check if we have unreceived Ack. If so, we \texttt{shrink} the node 
(see previous section) and add the un-confirmed message to the 
\texttt{linc\_wait} table, waiting for a new path to its recipient to 
be discovered. This way, when a node is disconnected, new path are build
around it in order to keep the connected properties of the graph.

\chapter{Data}
\section{Store Data}
When a user wants to store a file on LINC, it sends it to a node and
tells how many parts the file must be split into. Then, the node
cuts the file and makes pieces. The different pieces can be stored 
in different nodes. We do this in order to balance the load on the
nodes. This way, we can spread the load over all nodes and not overload
one node. A unique UUID is given to the user in order to recover the
file later on.

A table containing all files stored localy is built, named 
\texttt{linc\_files}. An entry in this table has the pattern 
\texttt{\{UUID, Part\_number, Total\_nb\_of\_parts\}}.
Files are written in the hard drive, not stored in the RAM. The files
are named \texttt{<UUID>.part<part\_nb>}

\section{Recover Data}
From any node, a user can ask for a given file with its UUID. Then the
User node will go into a blocking state, waiting for the full file to
come. Remember that the Agents must never be in blocking state, but user
nodes can.
When asking for a file, the user node broadcast to the whole network the
UUID of the file and to whom to send the parts. 
Each node will then check its stored files in \texttt{linc\_files} and
send all parts that match this UUID.

When the first part is received, the user can deduce the total number
of parts it is waiting for and will block until all pieces are recived.
There is a timeout to stop if a piece is missing or if the file is not
found on the network.

\section{Balance load}
When storage operations are done on a node, it automaticaly tries to
balance the load. It asks for the load of its direct neighbours. If 
a neighbour is less loaded than this node, we duplicate a part and send
it to the node. When the node successfuly received the part, it sends
back a message to the original node to delete the original part. We only
delete the file after it is properly stored on the new node. This is 
why the receiving node asks the origin node to delete it, and not the
origin node deleting it automaticaly.

With this mechanism, parts are spread across the network and the
load is even for all nodes.

When copying a part, there is a small chance that the part is not deleted
from the original node. This way we have multiple times the same part,
so if a node is lost, the data is still in another node. This duplication
is automaticaly done with small probability when copied to another node.

\section{Killed node and release data}
When a node is properly terminated, before shutting down, it sends all
its parts to an Agent neighbour. This way, no parts are lost.
 
Users and Agents can also send a signal to a node to release a given 
part or a given file. Then the nodes will delete the parts corresponding.

\chapter{Monitoring}
From a unique node, one can spawn Agents in remote nodes only knowing 
their name and cookie. All commands to do so are described in the 
Command chapter.

When using a shell to spawn the topology, distant nodes will display
debug informations in this shell.

A user can connect to the topology and ask for statistsics.
For now, a user can only ask for the load and the neighbours list of 
all nodes in the topology.

\chapter{Commands and Control}
See the Readme.md for cleaner explanations

\section{Setup and deploy}

\subsection{Prepare the processes}
If no distant shells are already running the command \texttt{make spawn}
will create detached processes ready to receive an Agent. One can also
give the \texttt{NB\_AGENT} variable to specify the number of agents to
spawn. For example \texttt{make spawn NB\_AGENT=15} will spawn 15 processes.
Processes spawned this way are named \texttt{n<id>@<host>} with 
\texttt{<id>} going from 1 to NB\_AGENT and \texttt{<host>} the local
host name.

All these names are using short names. In order to connect to a truely
remote node, change the Makefile and replace -sname with -name.
-sname is shorter and easyer to use for local demos.

\subsection{Spawn the Agents}
We then need to spawn the Agent code on these nodes. To do so, we use
the \texttt{setup} module. 
Enter \texttt{make ctrl} to open a control shell. It will be used to 
spawn the remote code and display the debug informations of the 
remote nodes. 

In the shell,
\begin{itemize}
\item \texttt{c(setup).} : Compile the module
\item \texttt{setup:spawn\_all(N).} : Auto spawn N nodes on remote nodes 
that were spawned with the Makefile (using the Makefile notation for
nodes name.). For example, after a  \texttt{make spawn NB\_AGENT=15}, 
use \texttt{setup:spawn\_all(15).} to bootstrap all nodes.
\item \texttt{setup:bootstrap(Node)} : Bootstrap a specific node
\item Now we must build a connected topology. For now, only one function
is available : \texttt{setup:build\_ring()} automaticaly builds a ring
on all spawned nodes. LINC does not need to be a ring but it is an
easy topology to start with and then transform it into a more complex
one.
\end{itemize}

Once all this is done, the control node can stay opened in order to 
display the debug.

\subsection{Spawn the user}
We must now spawn the user in a shell in order to interact with the
topology. \texttt{make user} starts a new shell.

The user node uses module \texttt{usr}.

In this shell,
\begin{itemize}
\item \texttt{c(usr).} : Compile the module
\item \texttt{usr:setup().} : Prepare the node, spawns listening processes
and tables so that this node can interact with the topology.
\item \texttt{usr:link(Node).} : Connects to the node Node.
\end{itemize}

Once this is done, the User node is ready to receive commands from humans.

\section{Topology control commands}
All commands can be sent to LINC by using \texttt{usr:sendCommand()} 
function. Once a node receives a commad, it spawns 
\texttt{command:<command>(<arguments>)}. So all possible commands are 
in the \texttt{command} module.

It can take several arguments :

\begin{itemize}
\item \texttt{usr:sendCommand(<command>, [<arguments>]).} Sends the command
with given arguments to the first node User is connected to.
\item \texttt{usr:sendCommand(<target>, <command>, [<arguments>]).} Same 
but this time the command is sent to and executed on node <target>.
If Target is \texttt{all}, the command will be spawned on all nodes.
\item \texttt{usr:shutdown(Node)} cleanly shutdowns an Agent. This
does not kill the erl process, only the Agent processes. This way
we can spawn back an Agent on the node.
\item \texttt{usr:sendCommand(<entry>, <target>, <command>, [<arguments>]).}
Same but specifies the node to be used as entry point on the topology.
\end{itemize}

There are a few topology commands that can be sent :
\begin{itemize}
\item \texttt{usr:sendCommand(Target, link, [Node])} : 
Add a link between the node executing this command and node Node.
\item \texttt{usr:neighbours()} : Asks all nodes to send their direct
neighbours to this User node. When the answer is received, it will
automaticaly be displayed in the user shell
\end{itemize}

\section{Data Management Commands}
Commands to manage data :
\begin{itemize}
\item \texttt{usr:store\_file(<file\_name>, <nb\_parts>} will store the file
and split it into parts. This returns \texttt{\{ok, UUID\}} if success.
\item \texttt{usr:recover\_file(<UUID>, <file\_name>} will gather the pieces
of the file and, once all are received, save the file under name 
<file\_name>
\item \texttt{usr:sendCommand(Target, balance, [])} : Will force the
balance of parts of node Target. (remember, Target can be \texttt{all})
\item \texttt{usr:load()} will ask all nodes to display their load on 
the user shell.
\end{itemize}

More commands are available, see the \texttt{command.erl} file for a 
complete list. 

\section{Cleaning}
After the tests, all files are stored in folder \texttt{nodes/}. 
To clean all files, just use \texttt{make clean}

Spawned processes are hard to kill because they are detached and have
no shell. Command \texttt{make kill} will kill all erlang processes.

\chapter{Conclusions and work left to do}
This topology is stable, works perfectly fine, is fully decentralized
with a random topology. But some improvements are still possible and
may be implemented in the future :

\begin{itemize}
\item Load balancing with stored size : For now, nodes balance the number
of parts stored, not the stored size. An improvement would be to consider
the size.
\item Crash resistance : When an node crashes, paths can be rebuilt with
the protocol described in previous chapters. But if too many nodes crash,
it is not recoverabl. I'm thinking about another protocol to handle this.
\item Distributed computations : It seems not that hard to distribute
computations on this network because spawned functions are spawned on 
different nodes, file parts can be replaced by data to treat or programs
to run. It is possible and could be fun to try.
\end{itemize}

\end{document}
